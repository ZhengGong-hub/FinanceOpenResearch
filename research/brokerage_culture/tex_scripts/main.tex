\documentclass[10pt]{article}
\usepackage{amsmath, amssymb, geometry, setspace, booktabs}
\geometry{margin=0.6in}
\usepackage{hyperref}
\hypersetup{colorlinks=true, linkcolor=blue, urlcolor=blue, citecolor=blue}
\usepackage{enumitem}
\setlist[itemize]{leftmargin=1.5em}
\setstretch{1.15}

% Title page
\title{Brokerage Culture}
\author{Zheng Gong}
\date{\today}

\begin{document}
\maketitle
\thispagestyle{empty}
\begin{center}
    \rule{0.8\textwidth}{0.4pt}
\end{center}
\vspace{1em}

%\tableofcontents
%\newpage

\section{Per-Question Analysis}

\subsection{Prompts}
You are analyzing an analyst question asked during a company earnings call. Your task is to assess the question across five cultural dimensions associated with the analyst’s employer (the brokerage firm), based only on the text of the question.

Evaluate the question on the following dimensions. For each dimension, provide:
\begin{itemize}
\item A score from 1 (very weak) to 5 (very strong)
\item A brief justification for the score
\end{itemize}

The five cultural dimensions are:

\begin{itemize}
\item \textbf{Innovation} — Does the question reflect forward thinking, creativity, or a novel approach to analysis?
\item \textbf{Integrity} — Does it seek transparency, challenge vague or evasive answers, or emphasize ethical concerns?
\item \textbf{Quality} — Is it precise, well-structured, and focused on operational excellence or customer satisfaction?
\item \textbf{Respect} — Is the tone professional, considerate, and reflective of mutual respect?
\item \textbf{Teamwork} — Does the question reference collaboration, consensus, shared goals, or coordination?
\end{itemize}

\textbf{Analyst Question:} is below


\subsection{Results}

% Main content starts here
\begin{tabular}{lrrrrr}
\toprule
 & innovation & integrity & quality & respect & teamwork \\
companyname &  &  &  &  &  \\
\midrule
JPMorgan Chase \& Co,  & 2.62 ± 0.65 & 3.52 ± 0.55 & 3.71 ± 0.54 & 4.1 ± 0.53 & 2.11 ± 0.47 \\
BofA Securities,  & 2.62 ± 0.64 & 3.52 ± 0.55 & 3.69 ± 0.55 & 4.08 ± 0.52 & 2.08 ± 0.44 \\
Goldman Sachs Group, Inc.,  & 2.74 ± 0.63 & 3.58 ± 0.52 & 3.81 ± 0.47 & 4.22 ± 0.54 & 2.13 ± 0.44 \\
Stifel, Nicolaus \& Company, Incorporated,  & 2.64 ± 0.65 & 3.48 ± 0.56 & 3.68 ± 0.55 & 4.08 ± 0.55 & 2.07 ± 0.44 \\
Jefferies LLC,  & 2.68 ± 0.65 & 3.48 ± 0.55 & 3.7 ± 0.54 & 4.12 ± 0.53 & 2.11 ± 0.46 \\
Wells Fargo Securities, LLC,  & 2.62 ± 0.64 & 3.49 ± 0.55 & 3.7 ± 0.53 & 4.11 ± 0.51 & 2.09 ± 0.44 \\
Deutsche Bank AG,  & 2.61 ± 0.64 & 3.49 ± 0.54 & 3.68 ± 0.56 & 4.08 ± 0.53 & 2.07 ± 0.44 \\
Raymond James \& Associates, Inc.,  & 2.6 ± 0.66 & 3.46 ± 0.56 & 3.66 ± 0.58 & 4.06 ± 0.55 & 2.06 ± 0.45 \\
Morgan Stanley,  & 2.75 ± 0.65 & 3.56 ± 0.53 & 3.78 ± 0.49 & 4.21 ± 0.54 & 2.13 ± 0.45 \\
Citigroup Inc.,  & 2.67 ± 0.64 & 3.56 ± 0.53 & 3.74 ± 0.51 & 4.12 ± 0.54 & 2.1 ± 0.44 \\
Barclays Bank PLC,  & 2.67 ± 0.64 & 3.52 ± 0.54 & 3.7 ± 0.55 & 4.11 ± 0.54 & 2.08 ± 0.43 \\
RBC Capital Markets,  & 2.66 ± 0.65 & 3.48 ± 0.56 & 3.7 ± 0.54 & 4.1 ± 0.52 & 2.1 ± 0.45 \\
Robert W. Baird \& Co. Incorporated,  & 2.6 ± 0.63 & 3.5 ± 0.54 & 3.72 ± 0.52 & 4.11 ± 0.52 & 2.08 ± 0.41 \\
KeyBanc Capital Markets Inc.,  & 2.57 ± 0.62 & 3.46 ± 0.56 & 3.67 ± 0.55 & 4.08 ± 0.52 & 2.06 ± 0.43 \\
Piper Sandler \& Co.,  & 2.67 ± 0.66 & 3.46 ± 0.56 & 3.71 ± 0.54 & 4.16 ± 0.52 & 2.11 ± 0.46 \\
Cr\'edit Suisse AG,  & 2.66 ± 0.64 & 3.53 ± 0.54 & 3.7 ± 0.54 & 4.09 ± 0.54 & 2.09 ± 0.45 \\
Sidoti \& Company, LLC & 2.44 ± 0.61 & 3.36 ± 0.56 & 3.5 ± 0.6 & 3.98 ± 0.49 & 1.99 ± 0.38 \\
UBS Investment Bank,  & 2.67 ± 0.63 & 3.56 ± 0.54 & 3.74 ± 0.53 & 4.13 ± 0.54 & 2.1 ± 0.44 \\
Truist Securities, Inc.,  & 2.62 ± 0.66 & 3.41 ± 0.56 & 3.62 ± 0.58 & 4.06 ± 0.52 & 2.08 ± 0.48 \\
TD Cowen,  & 2.7 ± 0.67 & 3.53 ± 0.55 & 3.74 ± 0.53 & 4.14 ± 0.56 & 2.12 ± 0.47 \\
William Blair \& Company L.L.C.,  & 2.69 ± 0.68 & 3.45 ± 0.56 & 3.68 ± 0.56 & 4.13 ± 0.55 & 2.11 ± 0.47 \\
Keefe, Bruyette, \& Woods, Inc.,  & 2.48 ± 0.6 & 3.44 ± 0.56 & 3.61 ± 0.6 & 4.04 ± 0.51 & 2.0 ± 0.42 \\
Stephens Inc.,  & 2.56 ± 0.62 & 3.46 ± 0.55 & 3.71 ± 0.52 & 4.14 ± 0.52 & 2.07 ± 0.41 \\
Evercore ISI Institutional Equities,  & 2.67 ± 0.65 & 3.54 ± 0.53 & 3.75 ± 0.51 & 4.15 ± 0.54 & 2.11 ± 0.47 \\
Oppenheimer \& Co. Inc.,  & 2.71 ± 0.67 & 3.46 ± 0.55 & 3.72 ± 0.52 & 4.14 ± 0.54 & 2.1 ± 0.46 \\
B. Riley Securities, Inc.,  & 2.61 ± 0.64 & 3.43 ± 0.56 & 3.68 ± 0.56 & 4.13 ± 0.51 & 2.08 ± 0.44 \\
D.A. Davidson \& Co.,  & 2.54 ± 0.64 & 3.36 ± 0.56 & 3.57 ± 0.6 & 4.01 ± 0.52 & 2.02 ± 0.42 \\
Needham \& Company, LLC,  & 2.66 ± 0.67 & 3.46 ± 0.55 & 3.69 ± 0.54 & 4.11 ± 0.53 & 2.1 ± 0.46 \\
Craig-Hallum Capital Group LLC,  & 2.66 ± 0.67 & 3.42 ± 0.56 & 3.64 ± 0.57 & 4.09 ± 0.53 & 2.09 ± 0.46 \\
ROTH Capital Partners, LLC,  & 2.64 ± 0.68 & 3.42 ± 0.58 & 3.64 ± 0.59 & 4.11 ± 0.53 & 2.07 ± 0.43 \\
BMO Capital Markets Equity Research & 2.6 ± 0.65 & 3.52 ± 0.55 & 3.7 ± 0.53 & 4.11 ± 0.53 & 2.1 ± 0.44 \\
Citizens JMP Securities, LLC,  & 2.72 ± 0.67 & 3.44 ± 0.59 & 3.65 ± 0.57 & 4.13 ± 0.54 & 2.11 ± 0.48 \\
Canaccord Genuity Corp.,  & 2.74 ± 0.67 & 3.42 ± 0.56 & 3.7 ± 0.55 & 4.13 ± 0.53 & 2.14 ± 0.51 \\
Wolfe Research, LLC & 2.65 ± 0.66 & 3.55 ± 0.56 & 3.66 ± 0.58 & 4.05 ± 0.57 & 2.09 ± 0.45 \\
Wedbush Securities Inc.,  & 2.61 ± 0.65 & 3.45 ± 0.56 & 3.67 ± 0.55 & 4.06 ± 0.52 & 2.07 ± 0.44 \\
CJS Securities, Inc. & 2.5 ± 0.61 & 3.43 ± 0.56 & 3.58 ± 0.61 & 4.05 ± 0.53 & 2.04 ± 0.43 \\
Janney Montgomery Scott LLC,  & 2.48 ± 0.65 & 3.37 ± 0.57 & 3.5 ± 0.62 & 3.96 ± 0.5 & 2.0 ± 0.45 \\
Macquarie Research & 2.62 ± 0.67 & 3.44 ± 0.56 & 3.64 ± 0.58 & 4.09 ± 0.54 & 2.06 ± 0.47 \\
Sandler O'Neill + Partners, L.P.,  & 2.44 ± 0.58 & 3.47 ± 0.54 & 3.58 ± 0.59 & 4.0 ± 0.5 & 2.02 ± 0.42 \\
FBR Capital Markets \& Co.,  & 2.53 ± 0.64 & 3.42 ± 0.56 & 3.61 ± 0.57 & 4.0 ± 0.5 & 2.03 ± 0.44 \\
Seaport Research Partners & 2.57 ± 0.65 & 3.42 ± 0.59 & 3.63 ± 0.59 & 4.04 ± 0.53 & 2.03 ± 0.43 \\
BB\&T Capital Markets,  & 2.42 ± 0.6 & 3.42 ± 0.57 & 3.54 ± 0.62 & 3.93 ± 0.54 & 1.98 ± 0.41 \\
Northland Capital Markets,  & 2.56 ± 0.66 & 3.32 ± 0.58 & 3.55 ± 0.6 & 4.02 ± 0.49 & 2.01 ± 0.4 \\
Guggenheim Securities, LLC,  & 2.83 ± 0.69 & 3.45 ± 0.57 & 3.69 ± 0.58 & 4.12 ± 0.52 & 2.15 ± 0.51 \\
Longbow Research LLC & 2.46 ± 0.59 & 3.45 ± 0.55 & 3.6 ± 0.6 & 4.0 ± 0.52 & 1.99 ± 0.37 \\
Barrington Research Associates, Inc.,  & 2.56 ± 0.7 & 3.35 ± 0.58 & 3.52 ± 0.62 & 4.02 ± 0.53 & 2.03 ± 0.47 \\
Susquehanna Financial Group, LLLP,  & 2.63 ± 0.66 & 3.49 ± 0.56 & 3.63 ± 0.59 & 4.02 ± 0.55 & 2.07 ± 0.47 \\
The Benchmark Company, LLC,  & 2.66 ± 0.67 & 3.4 ± 0.57 & 3.6 ± 0.61 & 4.05 ± 0.58 & 2.03 ± 0.46 \\
Sterne Agee \& Leach Inc.,  & 2.45 ± 0.58 & 3.4 ± 0.55 & 3.57 ± 0.61 & 3.91 ± 0.55 & 1.97 ± 0.39 \\
Sanford C. Bernstein \& Co., LLC.,  & 2.77 ± 0.66 & 3.63 ± 0.53 & 3.8 ± 0.54 & 4.12 ± 0.57 & 2.1 ± 0.46 \\
\bottomrule
\end{tabular}
    


\begin{tabular}{lrrrrr}
\toprule
 & innovation & integrity & quality & respect & teamwork \\
year &  &  &  &  &  \\
\midrule
2009 & 2.49 ± 0.61 & 3.44 ± 0.56 & 3.53 ± 0.61 & 3.91 ± 0.51 & 2.0 ± 0.43 \\
2010 & 2.51 ± 0.62 & 3.46 ± 0.55 & 3.6 ± 0.58 & 3.97 ± 0.53 & 2.01 ± 0.42 \\
2011 & 2.51 ± 0.63 & 3.43 ± 0.56 & 3.59 ± 0.59 & 3.98 ± 0.52 & 2.02 ± 0.43 \\
2012 & 2.5 ± 0.62 & 3.43 ± 0.56 & 3.57 ± 0.59 & 3.97 ± 0.52 & 2.02 ± 0.43 \\
2013 & 2.55 ± 0.64 & 3.43 ± 0.56 & 3.61 ± 0.59 & 4.0 ± 0.52 & 2.03 ± 0.44 \\
2014 & 2.58 ± 0.64 & 3.45 ± 0.56 & 3.64 ± 0.57 & 4.04 ± 0.52 & 2.05 ± 0.44 \\
2015 & 2.58 ± 0.64 & 3.47 ± 0.55 & 3.65 ± 0.56 & 4.05 ± 0.53 & 2.06 ± 0.44 \\
2016 & 2.59 ± 0.64 & 3.48 ± 0.55 & 3.66 ± 0.56 & 4.06 ± 0.52 & 2.06 ± 0.43 \\
2017 & 2.63 ± 0.65 & 3.48 ± 0.55 & 3.68 ± 0.55 & 4.07 ± 0.53 & 2.08 ± 0.44 \\
2018 & 2.64 ± 0.66 & 3.48 ± 0.56 & 3.69 ± 0.55 & 4.1 ± 0.52 & 2.09 ± 0.45 \\
2019 & 2.66 ± 0.66 & 3.49 ± 0.56 & 3.71 ± 0.54 & 4.13 ± 0.54 & 2.11 ± 0.46 \\
2020 & 2.73 ± 0.68 & 3.49 ± 0.56 & 3.72 ± 0.55 & 4.18 ± 0.55 & 2.11 ± 0.46 \\
2021 & 2.79 ± 0.67 & 3.48 ± 0.56 & 3.75 ± 0.53 & 4.23 ± 0.56 & 2.14 ± 0.47 \\
2022 & 2.73 ± 0.65 & 3.52 ± 0.56 & 3.74 ± 0.53 & 4.21 ± 0.55 & 2.12 ± 0.45 \\
2023 & 2.74 ± 0.64 & 3.52 ± 0.55 & 3.75 ± 0.52 & 4.21 ± 0.54 & 2.13 ± 0.45 \\
2024 & 2.76 ± 0.66 & 3.5 ± 0.56 & 3.76 ± 0.52 & 4.23 ± 0.54 & 2.13 ± 0.46 \\
\bottomrule
\end{tabular}


\section{Aggregated Questions Analysis}
Instead of analyzing individual questions, we aggregated questions by broker-quarter to achieve two key objectives:
\begin{itemize}
\item Increase variance between brokers - By combining multiple questions per broker, you aimed to capture more distinctive cultural patterns
\item Avoid unrepresentative short questions - Filtering out questions under 100 characters ensured more meaningful cultural signals
\item We have a longer context to work with the score, which is more representative of the broker's culture
\end{itemize}

The following results are based on samples of questions per broker-quarter, not the entire dataset.

\subsection{Prompts}
You are analyzing a collection of questions asked by different analysts on different earnings calls. Those analysts are employed by the same firm. Your task is to assess the questions in aggregate across five cultural dimensions associated with the analyst’s employer (the brokerage firm), based on the text of the question.

Evaluate the questions on the following dimensions. For each dimension, provide:
\begin{itemize}
\item A score from 1 (very weak) to 5 (very strong)
\item A brief justification for the score
\end{itemize}

The five cultural dimensions are:

\begin{itemize}
\item \textbf{Innovation} — Does the question reflect forward thinking, creativity, or a novel approach to analysis?
\item \textbf{Integrity} — Does it seek transparency, challenge vague or evasive answers, or emphasize ethical concerns?
\item \textbf{Quality} — Is it precise, well-structured, and focused on operational excellence or customer satisfaction?
\item \textbf{Respect} — Is the tone professional, considerate, and reflective of mutual respect?
\item \textbf{Teamwork} — Does the question reference collaboration, consensus, shared goals, or coordination?
\end{itemize}

You do not provide a score per question, rather, you provide an aggregated score based on those collections of questions per dimension.

Analyst Questions: is below.

\subsection{Results}

\begin{tabular}{lrrrrr}
\toprule
 & innovation & integrity & quality & respect & teamwork \\
companyname &  &  &  &  &  \\
\midrule
Goldman Sachs Group, Inc.,  & 3.0 ± 0.0 & 4.0 ± 0.0 & 4.03 ± 0.18 & 4.6 ± 0.49 & 2.02 ± 0.13 \\
Citigroup Inc.,  & 3.08 ± 0.27 & 4.02 ± 0.13 & 4.08 ± 0.27 & 4.6 ± 0.49 & 2.1 ± 0.3 \\
Stifel, Nicolaus \& Company, Incorporated,  & 3.1 ± 0.3 & 4.0 ± 0.0 & 4.1 ± 0.3 & 4.52 ± 0.5 & 2.13 ± 0.34 \\
Barclays Bank PLC,  & 3.1 ± 0.3 & 4.0 ± 0.0 & 4.1 ± 0.3 & 4.57 ± 0.5 & 2.11 ± 0.32 \\
BofA Securities,  & 3.02 ± 0.13 & 4.0 ± 0.0 & 4.02 ± 0.13 & 4.49 ± 0.5 & 2.08 ± 0.27 \\
Morgan Stanley,  & 3.05 ± 0.22 & 4.0 ± 0.0 & 4.05 ± 0.22 & 4.56 ± 0.5 & 2.06 ± 0.25 \\
JPMorgan Chase \& Co,  & 3.0 ± 0.0 & 4.0 ± 0.0 & 4.02 ± 0.13 & 4.44 ± 0.5 & 2.02 ± 0.13 \\
Deutsche Bank AG,  & 3.08 ± 0.27 & 3.98 ± 0.13 & 4.06 ± 0.25 & 4.54 ± 0.5 & 2.08 ± 0.27 \\
UBS Investment Bank,  & 3.03 ± 0.18 & 4.0 ± 0.0 & 4.03 ± 0.18 & 4.49 ± 0.5 & 2.05 ± 0.22 \\
William Blair \& Company L.L.C.,  & 3.19 ± 0.4 & 3.98 ± 0.13 & 4.18 ± 0.38 & 4.63 ± 0.49 & 2.23 ± 0.42 \\
Canaccord Genuity Corp.,  & 3.08 ± 0.28 & 4.0 ± 0.0 & 4.08 ± 0.28 & 4.66 ± 0.48 & 2.11 ± 0.32 \\
Macquarie Research & 3.16 ± 0.37 & 4.0 ± 0.0 & 4.16 ± 0.37 & 4.58 ± 0.5 & 2.19 ± 0.4 \\
Leerink Partners LLC,  & 3.13 ± 0.34 & 4.0 ± 0.0 & 4.13 ± 0.34 & 4.58 ± 0.5 & 2.14 ± 0.36 \\
Truist Securities, Inc.,  & 2.98 ± 0.13 & 4.0 ± 0.0 & 4.0 ± 0.0 & 4.48 ± 0.5 & 2.0 ± 0.0 \\
Craig-Hallum Capital Group LLC,  & 3.13 ± 0.34 & 4.0 ± 0.0 & 4.13 ± 0.34 & 4.55 ± 0.5 & 2.16 ± 0.37 \\
Susquehanna Financial Group, LLLP,  & 3.06 ± 0.25 & 4.0 ± 0.0 & 4.07 ± 0.25 & 4.47 ± 0.5 & 2.08 ± 0.28 \\
Janney Montgomery Scott LLC,  & 2.97 ± 0.25 & 3.97 ± 0.18 & 4.02 ± 0.13 & 4.34 ± 0.48 & 2.03 ± 0.18 \\
Citizens JMP Securities, LLC,  & 3.11 ± 0.32 & 4.0 ± 0.0 & 4.11 ± 0.32 & 4.57 ± 0.5 & 2.14 ± 0.36 \\
Keefe, Bruyette, \& Woods, Inc.,  & 2.74 ± 0.44 & 3.87 ± 0.34 & 4.05 ± 0.22 & 4.45 ± 0.5 & 2.02 ± 0.13 \\
Johnson Rice \& Company, L.L.C.,  & 3.1 ± 0.3 & 4.0 ± 0.0 & 4.11 ± 0.32 & 4.6 ± 0.5 & 2.14 ± 0.36 \\
Stephens Inc.,  & 2.97 ± 0.18 & 3.98 ± 0.13 & 4.02 ± 0.13 & 4.66 ± 0.48 & 2.03 ± 0.18 \\
Jefferies LLC,  & 3.03 ± 0.18 & 4.0 ± 0.0 & 4.05 ± 0.22 & 4.63 ± 0.49 & 2.08 ± 0.28 \\
Wedbush Securities Inc.,  & 3.05 ± 0.22 & 4.0 ± 0.0 & 4.05 ± 0.22 & 4.5 ± 0.5 & 2.06 ± 0.25 \\
TD Cowen,  & 2.98 ± 0.13 & 4.0 ± 0.0 & 4.0 ± 0.0 & 4.47 ± 0.5 & 2.08 ± 0.28 \\
KeyBanc Capital Markets Inc.,  & 2.98 ± 0.22 & 3.98 ± 0.13 & 4.02 ± 0.13 & 4.5 ± 0.5 & 2.06 ± 0.25 \\
Needham \& Company, LLC,  & 3.08 ± 0.28 & 4.0 ± 0.0 & 4.1 ± 0.3 & 4.58 ± 0.5 & 2.08 ± 0.28 \\
D.A. Davidson \& Co.,  & 2.98 ± 0.13 & 3.98 ± 0.13 & 4.0 ± 0.0 & 4.43 ± 0.5 & 2.02 ± 0.13 \\
Oppenheimer \& Co. Inc.,  & 3.11 ± 0.32 & 4.02 ± 0.13 & 4.11 ± 0.32 & 4.68 ± 0.47 & 2.16 ± 0.37 \\
RBC Capital Markets,  & 3.06 ± 0.25 & 4.0 ± 0.18 & 4.05 ± 0.22 & 4.6 ± 0.5 & 2.06 ± 0.25 \\
Longbow Research LLC & 2.98 ± 0.22 & 3.98 ± 0.13 & 4.02 ± 0.13 & 4.48 ± 0.5 & 2.05 ± 0.22 \\
B. Riley Securities, Inc.,  & 2.95 ± 0.22 & 3.98 ± 0.13 & 4.0 ± 0.0 & 4.57 ± 0.5 & 2.02 ± 0.13 \\
Raymond James \& Associates, Inc.,  & 3.02 ± 0.22 & 4.0 ± 0.0 & 4.03 ± 0.18 & 4.42 ± 0.5 & 2.05 ± 0.22 \\
ROTH Capital Partners, LLC,  & 3.08 ± 0.33 & 3.98 ± 0.13 & 4.11 ± 0.32 & 4.6 ± 0.5 & 2.11 ± 0.32 \\
Piper Sandler \& Co.,  & 2.98 ± 0.22 & 3.97 ± 0.18 & 4.02 ± 0.13 & 4.6 ± 0.5 & 2.06 ± 0.25 \\
Northland Capital Markets,  & 2.97 ± 0.31 & 3.95 ± 0.22 & 4.03 ± 0.18 & 4.53 ± 0.5 & 2.03 ± 0.18 \\
Robert W. Baird \& Co. Incorporated,  & 3.05 ± 0.22 & 4.0 ± 0.0 & 4.05 ± 0.22 & 4.58 ± 0.5 & 2.08 ± 0.28 \\
CJS Securities, Inc. & 2.95 ± 0.22 & 3.98 ± 0.13 & 4.0 ± 0.0 & 4.63 ± 0.49 & 2.0 ± 0.0 \\
Sidoti \& Company, LLC & 3.0 ± 0.18 & 4.0 ± 0.0 & 4.02 ± 0.13 & 4.45 ± 0.5 & 2.02 ± 0.13 \\
Barrington Research Associates, Inc.,  & 3.05 ± 0.22 & 4.0 ± 0.0 & 4.05 ± 0.22 & 4.62 ± 0.49 & 2.08 ± 0.28 \\
Maxim Group LLC,  & 3.07 ± 0.25 & 4.0 ± 0.0 & 4.07 ± 0.25 & 4.54 ± 0.5 & 2.07 ± 0.25 \\
Sanford C. Bernstein \& Co., LLC.,  & 3.18 ± 0.39 & 3.98 ± 0.13 & 4.21 ± 0.41 & 4.47 ± 0.5 & 2.16 ± 0.37 \\
The Benchmark Company, LLC,  & 3.12 ± 0.37 & 3.97 ± 0.18 & 4.1 ± 0.3 & 4.46 ± 0.5 & 2.16 ± 0.37 \\
NOBLE Capital Markets, Inc.,  & 3.0 ± 0.0 & 4.0 ± 0.0 & 4.0 ± 0.0 & 4.43 ± 0.5 & 2.03 ± 0.18 \\
Ladenburg Thalmann \& Co. Inc.,  & 3.0 ± 0.18 & 3.98 ± 0.13 & 4.02 ± 0.13 & 4.43 ± 0.5 & 2.07 ± 0.25 \\
Wells Fargo Securities, LLC,  & 3.05 ± 0.22 & 4.0 ± 0.0 & 4.07 ± 0.25 & 4.6 ± 0.49 & 2.05 ± 0.22 \\
Capital One Securities, Inc.,  & 2.9 ± 0.4 & 3.93 ± 0.31 & 4.03 ± 0.18 & 4.45 ± 0.5 & 2.08 ± 0.28 \\
Northcoast Research Partners, LLC & 3.02 ± 0.23 & 4.0 ± 0.0 & 4.05 ± 0.22 & 4.64 ± 0.48 & 2.03 ± 0.18 \\
Cr\'edit Suisse AG,  & 3.0 ± 0.19 & 4.0 ± 0.0 & 4.02 ± 0.13 & 4.41 ± 0.5 & 2.05 ± 0.22 \\
Seaport Research Partners & 2.98 ± 0.13 & 4.0 ± 0.0 & 4.02 ± 0.13 & 4.5 ± 0.5 & 2.02 ± 0.13 \\
Tudor, Pickering, Holt \& Co. Securities, LLC,  & 3.07 ± 0.26 & 3.98 ± 0.13 & 4.07 ± 0.26 & 4.63 ± 0.49 & 2.05 ± 0.22 \\
\bottomrule
\end{tabular}

\newpage

\begin{tabular}{lrrrrr}
\toprule
 & innovation & integrity & quality & respect & teamwork \\
companyname &  &  &  &  &  \\
\midrule
Evercore ISI Institutional Equities,  & 3.05 ± 0.3 & 4.0 ± 0.0 & 4.07 ± 0.26 & 4.54 ± 0.5 & 2.07 ± 0.26 \\
Guggenheim Securities, LLC,  & 3.12 ± 0.33 & 4.0 ± 0.0 & 4.12 ± 0.33 & 4.52 ± 0.5 & 2.12 ± 0.33 \\
Cantor Fitzgerald \& Co.,  & 3.24 ± 0.47 & 4.0 ± 0.0 & 4.26 ± 0.44 & 4.71 ± 0.46 & 2.27 ± 0.45 \\
Wolfe Research, LLC & 3.11 ± 0.32 & 4.0 ± 0.0 & 4.11 ± 0.32 & 4.43 ± 0.5 & 2.13 ± 0.34 \\
Mizuho Securities USA LLC,  & 3.11 ± 0.32 & 4.0 ± 0.0 & 4.11 ± 0.32 & 4.48 ± 0.5 & 2.13 ± 0.34 \\
Green Street Advisors, LLC,  & 3.0 ± 0.28 & 4.0 ± 0.0 & 4.04 ± 0.19 & 4.6 ± 0.49 & 2.04 ± 0.19 \\
Thompson Research Group, LLC & 3.08 ± 0.33 & 4.02 ± 0.14 & 4.15 ± 0.36 & 4.85 ± 0.36 & 2.09 ± 0.3 \\
BTIG, LLC,  & 3.21 ± 0.41 & 3.98 ± 0.14 & 4.19 ± 0.4 & 4.51 ± 0.5 & 2.23 ± 0.42 \\
Morningstar Inc.,  & 3.11 ± 0.32 & 4.0 ± 0.0 & 4.11 ± 0.32 & 4.6 ± 0.49 & 2.11 ± 0.32 \\
Vertical Research Partners, LLC & 3.1 ± 0.3 & 3.98 ± 0.14 & 4.1 ± 0.3 & 4.6 ± 0.5 & 2.12 ± 0.32 \\
Zelman \& Associates LLC & 3.06 ± 0.24 & 4.0 ± 0.0 & 4.08 ± 0.27 & 4.62 ± 0.49 & 2.08 ± 0.27 \\
Telsey Advisory Group LLC & 3.19 ± 0.4 & 3.92 ± 0.27 & 4.12 ± 0.32 & 4.96 ± 0.19 & 2.12 ± 0.32 \\
Colliers Securities LLC,  & 3.0 ± 0.28 & 3.98 ± 0.14 & 4.04 ± 0.2 & 4.61 ± 0.49 & 2.08 ± 0.27 \\
Cleveland Research Company LLC & 3.02 ± 0.14 & 4.0 ± 0.0 & 4.04 ± 0.2 & 4.65 ± 0.48 & 2.04 ± 0.2 \\
Compass Point Research \& Trading, LLC,  & 2.86 ± 0.4 & 3.92 ± 0.27 & 4.06 ± 0.24 & 4.32 ± 0.47 & 2.04 ± 0.2 \\
Tuohy Brothers Investment Research, Inc. & 3.52 ± 0.5 & 3.84 ± 0.37 & 4.36 ± 0.48 & 4.62 ± 0.49 & 2.42 ± 0.5 \\
BMO Capital Markets Equity Research & 3.0 ± 0.2 & 4.0 ± 0.0 & 4.04 ± 0.2 & 4.59 ± 0.5 & 2.08 ± 0.28 \\
MKM Partners LLC,  & 3.1 ± 0.31 & 3.96 ± 0.2 & 4.06 ± 0.24 & 4.4 ± 0.49 & 2.06 ± 0.24 \\
CL King \& Associates, Inc.,  & 3.08 ± 0.35 & 4.0 ± 0.0 & 4.1 ± 0.31 & 4.69 ± 0.47 & 2.12 ± 0.33 \\
Lake Street Capital Markets, LLC,  & 3.06 ± 0.25 & 4.0 ± 0.0 & 4.07 ± 0.25 & 4.72 ± 0.46 & 2.06 ± 0.25 \\
Imperial Capital, LLC,  & 3.17 ± 0.44 & 3.91 ± 0.28 & 4.11 ± 0.32 & 4.52 ± 0.5 & 2.2 ± 0.4 \\
G.research, LLC & 3.07 ± 0.25 & 3.98 ± 0.15 & 4.07 ± 0.25 & 4.44 ± 0.5 & 2.04 ± 0.21 \\
Simmons \& Company International,  & 3.09 ± 0.29 & 4.0 ± 0.0 & 4.11 ± 0.32 & 4.39 ± 0.49 & 2.09 ± 0.29 \\
MoffettNathanson LLC & 3.52 ± 0.5 & 4.02 ± 0.15 & 4.52 ± 0.5 & 4.81 ± 0.4 & 2.52 ± 0.5 \\
Tieton Capital Management, LLC & 3.05 ± 0.22 & 4.0 ± 0.0 & 4.05 ± 0.22 & 4.74 ± 0.44 & 2.05 ± 0.22 \\
First Analysis Securities Corporation,  & 3.1 ± 0.3 & 4.0 ± 0.22 & 4.07 ± 0.26 & 4.48 ± 0.5 & 2.1 ± 0.3 \\
Buckingham Research Group Incorporated & 3.0 ± 0.22 & 4.0 ± 0.0 & 4.02 ± 0.15 & 4.29 ± 0.46 & 2.02 ± 0.15 \\
Huber Research Partners, LLC & 2.66 ± 0.53 & 3.9 ± 0.3 & 4.02 ± 0.16 & 4.12 ± 0.4 & 2.05 ± 0.22 \\
Cross Research LLC & 3.08 ± 0.27 & 4.0 ± 0.0 & 4.08 ± 0.27 & 4.53 ± 0.51 & 2.1 ± 0.3 \\
Stonegate Capital Markets, Inc.,  & 2.82 ± 0.38 & 3.92 ± 0.27 & 4.0 ± 0.0 & 4.55 ± 0.5 & 2.08 ± 0.27 \\
Kansas City Capital Associates & 3.05 ± 0.22 & 3.97 ± 0.16 & 4.03 ± 0.16 & 4.28 ± 0.46 & 2.03 ± 0.16 \\
Boenning and Scattergood, Inc.,  & 3.05 ± 0.22 & 4.0 ± 0.0 & 4.05 ± 0.22 & 4.54 ± 0.5 & 2.05 ± 0.22 \\
Glenrock Associates LLC & 2.8 ± 0.41 & 3.97 ± 0.16 & 3.87 ± 0.34 & 4.0 ± 0.23 & 2.08 ± 0.27 \\
Bernstein Autonomous LLP & 3.08 ± 0.27 & 4.03 ± 0.16 & 4.2 ± 0.41 & 4.41 ± 0.5 & 2.13 ± 0.34 \\
H.C. Wainwright \& Co, LLC,  & 3.18 ± 0.39 & 4.0 ± 0.0 & 4.18 ± 0.39 & 4.62 ± 0.49 & 2.18 ± 0.39 \\
Nomura Securities Co. Ltd.,  & 3.05 ± 0.23 & 4.0 ± 0.0 & 4.08 ± 0.27 & 4.61 ± 0.5 & 2.08 ± 0.27 \\
Scotia Howard Weil,  & 3.03 ± 0.16 & 4.0 ± 0.0 & 4.03 ± 0.16 & 4.45 ± 0.5 & 2.05 ± 0.23 \\
Sandler O'Neill + Partners, L.P.,  & 2.9 ± 0.39 & 4.0 ± 0.0 & 4.03 ± 0.16 & 4.45 ± 0.5 & 2.03 ± 0.16 \\
Alembic Global Advisors & 3.03 ± 0.17 & 4.0 ± 0.0 & 4.08 ± 0.28 & 4.42 ± 0.5 & 2.08 ± 0.28 \\
FIG Partners, LLC,  & 2.47 ± 0.51 & 3.56 ± 0.5 & 4.0 ± 0.0 & 4.36 ± 0.49 & 2.0 ± 0.0 \\
Hovde Group, LLC,  & 2.97 ± 0.3 & 3.97 ± 0.17 & 4.03 ± 0.17 & 4.69 ± 0.47 & 2.03 ± 0.17 \\
FBR Capital Markets \& Co.,  & 3.03 ± 0.17 & 4.0 ± 0.0 & 4.03 ± 0.17 & 4.54 ± 0.5 & 2.09 ± 0.28 \\
Singular Research, LLC & 3.0 ± 0.25 & 3.97 ± 0.17 & 4.03 ± 0.17 & 4.36 ± 0.49 & 2.03 ± 0.17 \\
Pacific Crest Securities, Inc.,  & 3.06 ± 0.25 & 4.0 ± 0.0 & 4.06 ± 0.25 & 4.41 ± 0.5 & 2.12 ± 0.34 \\
Loop Capital Markets LLC,  & 3.09 ± 0.3 & 3.97 ± 0.18 & 4.06 ± 0.25 & 4.72 ± 0.46 & 2.06 ± 0.25 \\
Avondale Partners, LLC,  & 3.0 ± 0.0 & 4.0 ± 0.0 & 4.0 ± 0.0 & 4.25 ± 0.44 & 2.0 ± 0.0 \\
Feltl and Company, Inc.,  & 2.97 ± 0.18 & 3.97 ± 0.18 & 4.0 ± 0.0 & 4.39 ± 0.5 & 2.03 ± 0.18 \\
Brean Capital, LLC,  & 3.03 ± 0.18 & 3.97 ± 0.18 & 4.0 ± 0.0 & 4.32 ± 0.48 & 2.13 ± 0.34 \\
Wunderlich Securities Inc.,  & 3.06 ± 0.25 & 3.97 ± 0.18 & 4.03 ± 0.18 & 4.29 ± 0.46 & 2.1 ± 0.3 \\
BB\&T Capital Markets,  & 3.0 ± 0.0 & 4.0 ± 0.0 & 4.0 ± 0.0 & 4.3 ± 0.47 & 2.1 ± 0.3 \\
\bottomrule
\end{tabular}

\newpage

\begin{tabular}{lrrrrr}
\toprule
 & innovation & integrity & quality & respect & teamwork \\
companyname &  &  &  &  &  \\
\midrule
Sterne Agee \& Leach Inc.,  & 2.97 ± 0.19 & 4.0 ± 0.0 & 4.0 ± 0.0 & 4.17 ± 0.38 & 2.0 ± 0.0 \\
Zacks Investment Research, Inc. & 3.03 ± 0.19 & 4.0 ± 0.0 & 4.03 ± 0.19 & 4.21 ± 0.41 & 2.1 ± 0.31 \\
Melius Research LLC & 3.39 ± 0.5 & 4.0 ± 0.27 & 4.32 ± 0.48 & 4.71 ± 0.46 & 2.43 ± 0.5 \\
Scotiabank Global Banking and Markets,  & 3.0 ± 0.0 & 4.0 ± 0.0 & 4.0 ± 0.0 & 4.46 ± 0.51 & 2.07 ± 0.26 \\
Griffin Securities, Inc.,  & 3.07 ± 0.26 & 4.0 ± 0.0 & 4.07 ± 0.26 & 4.29 ± 0.46 & 2.07 ± 0.26 \\
Consumer Edge Research, LLC & 3.21 ± 0.42 & 4.0 ± 0.0 & 4.21 ± 0.42 & 4.46 ± 0.51 & 2.25 ± 0.44 \\
Nephron Research LLC & 3.15 ± 0.36 & 4.0 ± 0.0 & 4.15 ± 0.36 & 4.74 ± 0.45 & 2.18 ± 0.4 \\
Pickering Energy Partners Insights & 3.15 ± 0.37 & 4.0 ± 0.0 & 4.15 ± 0.37 & 4.35 ± 0.48 & 2.23 ± 0.43 \\
Pivotal Research Group LLC & 3.0 ± 0.0 & 4.0 ± 0.0 & 4.08 ± 0.27 & 4.46 ± 0.51 & 2.04 ± 0.2 \\
Hilliard Lyons,  & 2.76 ± 0.44 & 3.96 ± 0.2 & 4.0 ± 0.0 & 4.32 ± 0.48 & 2.0 ± 0.0 \\
Alliance Global Partners,  & 3.08 ± 0.28 & 4.0 ± 0.0 & 4.08 ± 0.28 & 4.6 ± 0.5 & 2.08 ± 0.28 \\
Gordon Haskett Research Advisors & 3.04 ± 0.2 & 3.96 ± 0.2 & 4.0 ± 0.0 & 4.52 ± 0.51 & 2.0 ± 0.0 \\
CRT Capital Group LLC,  & 3.0 ± 0.29 & 4.0 ± 0.0 & 4.04 ± 0.2 & 4.32 ± 0.48 & 2.04 ± 0.2 \\
Drexel Hamilton, LLC,  & 3.0 ± 0.0 & 4.0 ± 0.0 & 4.0 ± 0.0 & 4.12 ± 0.34 & 2.04 ± 0.2 \\
Morgan Dempsey Capital Management LLC & 3.12 ± 0.34 & 3.88 ± 0.34 & 4.04 ± 0.2 & 4.17 ± 0.38 & 2.04 ± 0.2 \\
BMO Capital Markets U.S. & 3.0 ± 0.0 & 4.0 ± 0.0 & 4.0 ± 0.0 & 4.17 ± 0.38 & 2.04 ± 0.2 \\
Joh. Berenberg, Gossler \& Co. KG,  & 3.14 ± 0.35 & 4.0 ± 0.0 & 4.14 ± 0.35 & 4.59 ± 0.5 & 2.18 ± 0.4 \\
BWS Financial Inc. & 2.95 ± 0.22 & 4.0 ± 0.0 & 4.0 ± 0.0 & 4.0 ± 0.0 & 2.0 ± 0.0 \\
CLSA Limited,  & 3.05 ± 0.22 & 4.0 ± 0.0 & 4.05 ± 0.22 & 4.29 ± 0.46 & 2.1 ± 0.3 \\
\bottomrule
\end{tabular}


\end{document}